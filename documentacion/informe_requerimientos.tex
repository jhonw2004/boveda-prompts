\documentclass[12pt,letterpaper]{article}

% Codificacion
\usepackage[utf8]{inputenc}
\usepackage[T1]{fontenc}
\usepackage[spanish]{babel}

% Margenes APA
\usepackage[letterpaper,margin=2.54cm]{geometry}

% Interlineado APA
\usepackage{setspace}
\doublespacing

% Sangria
\setlength{\parindent}{1.27cm}

% Encabezados
\usepackage{fancyhdr}
\pagestyle{fancy}
\fancyhf{}
\rhead{\thepage}
\renewcommand{\headrulewidth}{0pt}

% Tablas y figuras
\usepackage{graphicx}
\usepackage{float}
\usepackage{tabularx}
\usepackage{array}
\usepackage{longtable}

% Colores
\usepackage[table]{xcolor}
\definecolor{primaryBlue}{RGB}{41,98,255}

% Listas
\usepackage{enumitem}
\setlist[itemize]{leftmargin=1.27cm}
\setlist[enumerate]{leftmargin=1.27cm}

% Hipervinculos
\usepackage[hidelinks]{hyperref}

% ============================================================================

\begin{document}

% ===================== PORTADA =====================
\begin{titlepage}
    \centering
    
    {\fontsize{26}{32}\selectfont \bfseries Universidad Privada Domingo Savio}\\[1cm]
    
    \includegraphics[width=0.3\textwidth]{upds_logo.png}\\[1cm]
    
    {\Huge \bfseries Proyecto Formativo}\\[0.5cm]
    {\LARGE \bfseries Bóveda de Prompts}\\[0.8cm]
    
    \Large
    \textbf{Integrantes:}\\[0.2cm]
    Jhon Walter Choque Condori\\
    Maycol Robinm Quispe Calani\\
    Jose Carlos Peña Pedraza\\[0.6cm]
        
    \textbf{Docente:}\\[0.2cm]
    Lic. [Nombre del Docente]\\[0.6cm]
        
    \textbf{Asignatura:}\\[0.2cm]
    Ingeniería de Software
    
    \vfill
    {\Large \bfseries Enero 2026}
\end{titlepage}

% ===================== INDICE =====================
\tableofcontents
\newpage

% ===================== CONTENIDO =====================

\section{Introducción}

\subsection{Contexto del Proyecto}

En la era actual de la inteligencia artificial generativa, los profesionales, desarrolladores, investigadores y entusiastas de la tecnología utilizan constantemente prompts para interactuar con modelos de lenguaje como ChatGPT, Claude, Gemini, entre otros. Estos prompts representan instrucciones cuidadosamente elaboradas que permiten obtener resultados óptimos de los sistemas de IA.

Sin embargo, la gestión eficiente de estos prompts se ha convertido en un desafío significativo. Los usuarios frecuentemente enfrentan problemas como la pérdida de prompts valiosos, la dificultad para organizar y categorizar grandes colecciones, y la incapacidad de acceder rápidamente a prompts específicos cuando los necesitan.

\subsection{Descripción del Problema}

Los principales problemas identificados que motivaron el desarrollo de este proyecto son:

\begin{itemize}
    \item \textbf{Desorganización}: Los usuarios almacenan sus prompts en archivos de texto dispersos, notas digitales, documentos de Word o simplemente confían en su memoria.
    
    \item \textbf{Falta de accesibilidad}: No existe un repositorio centralizado que permita acceder a los prompts desde cualquier dispositivo con conexión a internet.
    
    \item \textbf{Pérdida de información}: Los prompts valiosos se pierden cuando se cambian de dispositivo, se formatean equipos o se eliminan accidentalmente.
    
    \item \textbf{Ausencia de categorización}: No hay un sistema estructurado que permita clasificar y buscar prompts por categorías, etiquetas o palabras clave.
    
    \item \textbf{Falta de exportación}: No existe una forma estandarizada de respaldar o compartir colecciones de prompts.
\end{itemize}

\subsection{Objetivo General}

Desarrollar un sistema web completo denominado ``Bóveda de Prompts'' que permita a los usuarios gestionar, almacenar, organizar y exportar su colección personal de prompts de inteligencia artificial de manera segura, eficiente y accesible desde cualquier dispositivo.

\subsection{Objetivos Específicos}

\begin{enumerate}
    \item Implementar un sistema de autenticación robusto con verificación por correo electrónico y gestión segura de sesiones mediante tokens JWT.
    
    \item Desarrollar funcionalidades CRUD completas para la gestión de prompts, incluyendo creación, lectura, actualización y eliminación.
    
    \item Crear un sistema de categorización mediante etiquetas que permita organizar los prompts de manera flexible.
    
    \item Implementar capacidades de búsqueda y filtrado avanzado para localizar prompts de forma rápida y precisa.
    
    \item Desarrollar un módulo de exportación que soporte múltiples formatos (JSON, Markdown, CSV, TXT).
    
    \item Diseñar una interfaz de usuario moderna, intuitiva y responsiva que proporcione una experiencia de usuario óptima.
\end{enumerate}

\subsection{Alcance del Proyecto}

El proyecto ``Bóveda de Prompts'' contempla el desarrollo de una aplicación web de pila completa (full-stack) que incluye:

\begin{itemize}
    \item \textbf{Frontend}: Aplicación web de una sola página (SPA) desarrollada con React 18.
    \item \textbf{Backend}: API RESTful desarrollada con Node.js y Express.js.
    \item \textbf{Base de datos}: Sistema de gestión de base de datos relacional PostgreSQL.
    \item \textbf{Sistema de correos}: Integración con servicios SMTP para verificación de usuarios.
\end{itemize}

El sistema está diseñado para uso personal, donde cada usuario tiene acceso exclusivo a su propia colección de prompts, garantizando la privacidad y seguridad de la información.

% ============================================================================
\section{Requerimientos Funcionales}

Los requerimientos funcionales describen las funcionalidades específicas que el sistema debe proporcionar. Se han organizado por módulos para facilitar su comprensión e implementación.

\subsection{Módulo de Autenticación (RF-AUTH)}

\begin{table}[H]
\centering
\small
\begin{tabular}{|c|p{9.5cm}|c|}
\hline
\rowcolor{primaryBlue!20}
\textbf{ID} & \textbf{Descripción} & \textbf{Prioridad} \\
\hline
RF-AUTH-01 & El sistema debe permitir el registro de nuevos usuarios mediante correo electrónico, contraseña y nombre de usuario. & Alta \\
\hline
RF-AUTH-02 & El sistema debe validar que el correo electrónico sea único en la base de datos antes de completar el registro. & Alta \\
\hline
RF-AUTH-03 & El sistema debe enviar un correo de verificación automáticamente al registrarse un nuevo usuario. & Alta \\
\hline
RF-AUTH-04 & El sistema debe permitir la verificación del correo electrónico mediante un token único enviado al usuario. & Alta \\
\hline
RF-AUTH-05 & El sistema debe permitir el reenvío del correo de verificación si el usuario no lo recibió. & Media \\
\hline
RF-AUTH-06 & El sistema debe permitir el inicio de sesión mediante correo electrónico y contraseña. & Alta \\
\hline
RF-AUTH-07 & El sistema debe generar un token JWT al iniciar sesión exitosamente. & Alta \\
\hline
RF-AUTH-08 & El sistema debe validar que el usuario haya verificado su correo antes de permitir el acceso completo. & Alta \\
\hline
RF-AUTH-09 & El sistema debe permitir al usuario cerrar sesión, invalidando el token activo. & Alta \\
\hline
RF-AUTH-10 & El sistema debe mostrar la información del usuario autenticado actualmente. & Media \\
\hline
\end{tabular}
\caption{Requerimientos Funcionales del Módulo de Autenticación}
\end{table}

\subsection{Módulo de Gestión de Prompts (RF-PROMPT)}

\begin{table}[H]
\centering
\small
\begin{tabular}{|c|p{9.5cm}|c|}
\hline
\rowcolor{primaryBlue!20}
\textbf{ID} & \textbf{Descripción} & \textbf{Prioridad} \\
\hline
RF-PROMPT-01 & El sistema debe permitir crear nuevos prompts con título, contenido, descripción opcional y etiquetas. & Alta \\
\hline
RF-PROMPT-02 & El sistema debe permitir visualizar la lista de todos los prompts del usuario con paginación. & Alta \\
\hline
RF-PROMPT-03 & El sistema debe permitir ver el detalle completo de un prompt específico. & Alta \\
\hline
RF-PROMPT-04 & El sistema debe permitir editar un prompt existente (título, contenido, descripción, etiquetas). & Alta \\
\hline
RF-PROMPT-05 & El sistema debe permitir eliminar un prompt existente. & Alta \\
\hline
RF-PROMPT-06 & El sistema debe registrar automáticamente la fecha de creación y última modificación de cada prompt. & Media \\
\hline
RF-PROMPT-07 & El sistema debe limitar el número máximo de prompts por usuario a 1000. & Media \\
\hline
RF-PROMPT-08 & El sistema debe limitar el número máximo de etiquetas por prompt a 10. & Media \\
\hline
RF-PROMPT-09 & El sistema debe implementar control de concurrencia optimista para actualizaciones. & Media \\
\hline
\end{tabular}
\caption{Requerimientos Funcionales del Módulo de Gestión de Prompts}
\end{table}

\subsection{Módulo de Búsqueda y Filtrado (RF-SEARCH)}

\begin{table}[H]
\centering
\small
\begin{tabular}{|c|p{9.5cm}|c|}
\hline
\rowcolor{primaryBlue!20}
\textbf{ID} & \textbf{Descripción} & \textbf{Prioridad} \\
\hline
RF-SEARCH-01 & El sistema debe permitir buscar prompts por palabra clave (búsqueda full-text). & Alta \\
\hline
RF-SEARCH-02 & El sistema debe permitir filtrar prompts por etiquetas específicas. & Alta \\
\hline
RF-SEARCH-03 & El sistema debe permitir ordenar los prompts por fecha de creación o modificación. & Media \\
\hline
RF-SEARCH-04 & El sistema debe permitir ordenar los prompts de forma ascendente o descendente. & Media \\
\hline
RF-SEARCH-05 & El sistema debe implementar paginación con límite configurable de resultados por página. & Alta \\
\hline
\end{tabular}
\caption{Requerimientos Funcionales del Módulo de Búsqueda y Filtrado}
\end{table}

\subsection{Módulo de Exportación (RF-EXPORT)}

\begin{table}[H]
\centering
\small
\begin{tabular}{|c|p{9.5cm}|c|}
\hline
\rowcolor{primaryBlue!20}
\textbf{ID} & \textbf{Descripción} & \textbf{Prioridad} \\
\hline
RF-EXPORT-01 & El sistema debe permitir exportar prompts en formato JSON. & Alta \\
\hline
RF-EXPORT-02 & El sistema debe permitir exportar prompts en formato Markdown. & Media \\
\hline
RF-EXPORT-03 & El sistema debe permitir exportar prompts en formato CSV. & Media \\
\hline
RF-EXPORT-04 & El sistema debe permitir exportar prompts en formato TXT. & Baja \\
\hline
RF-EXPORT-05 & El sistema debe permitir exportar la colección completa o prompts seleccionados. & Alta \\
\hline
RF-EXPORT-06 & El sistema debe limitar la exportación a un máximo de 500 prompts por operación. & Media \\
\hline
\end{tabular}
\caption{Requerimientos Funcionales del Módulo de Exportación}
\end{table}

\subsection{Módulo de Estadísticas (RF-STATS)}

\begin{table}[H]
\centering
\small
\begin{tabular}{|c|p{9.5cm}|c|}
\hline
\rowcolor{primaryBlue!20}
\textbf{ID} & \textbf{Descripción} & \textbf{Prioridad} \\
\hline
RF-STATS-01 & El sistema debe mostrar el número total de prompts almacenados por el usuario. & Media \\
\hline
RF-STATS-02 & El sistema debe mostrar las etiquetas más utilizadas. & Baja \\
\hline
RF-STATS-03 & El sistema debe mostrar estadísticas de uso reciente. & Baja \\
\hline
\end{tabular}
\caption{Requerimientos Funcionales del Módulo de Estadísticas}
\end{table}

% ============================================================================
\section{Requerimientos No Funcionales}

Los requerimientos no funcionales definen las características de calidad que debe cumplir el sistema.

\subsection{Seguridad (RNF-SEC)}

\begin{table}[H]
\centering
\small
\begin{tabular}{|c|p{9.5cm}|c|}
\hline
\rowcolor{primaryBlue!20}
\textbf{ID} & \textbf{Descripción} & \textbf{Prioridad} \\
\hline
RNF-SEC-01 & Las contraseñas deben almacenarse utilizando algoritmo bcrypt con factor de costo mínimo de 12. & Alta \\
\hline
RNF-SEC-02 & Los tokens JWT deben tener una expiración configurable (por defecto 7 días). & Alta \\
\hline
RNF-SEC-03 & Todas las consultas a la base de datos deben usar queries parametrizados para prevenir inyección SQL. & Alta \\
\hline
RNF-SEC-04 & El sistema debe implementar rate limiting: máximo 5 intentos de login en 15 minutos por IP. & Alta \\
\hline
RNF-SEC-05 & El sistema debe implementar headers de seguridad HTTP mediante Helmet.js. & Alta \\
\hline
RNF-SEC-06 & Las comunicaciones entre cliente y servidor deben realizarse exclusivamente sobre HTTPS en producción. & Alta \\
\hline
RNF-SEC-07 & El sistema debe implementar CORS con orígenes permitidos configurables. & Media \\
\hline
\end{tabular}
\caption{Requerimientos No Funcionales de Seguridad}
\end{table}

\subsection{Rendimiento (RNF-PERF)}

\begin{table}[H]
\centering
\small
\begin{tabular}{|c|p{9.5cm}|c|}
\hline
\rowcolor{primaryBlue!20}
\textbf{ID} & \textbf{Descripción} & \textbf{Prioridad} \\
\hline
RNF-PERF-01 & El tiempo de respuesta de las consultas API debe ser menor a 500ms en el percentil 95. & Alta \\
\hline
RNF-PERF-02 & La base de datos debe implementar índices optimizados en campos de búsqueda frecuente. & Alta \\
\hline
RNF-PERF-03 & El sistema debe soportar al menos 100 usuarios concurrentes sin degradación significativa. & Media \\
\hline
RNF-PERF-04 & La carga inicial de la aplicación web debe ser menor a 3 segundos en conexiones estándar. & Media \\
\hline
\end{tabular}
\caption{Requerimientos No Funcionales de Rendimiento}
\end{table}

\subsection{Usabilidad (RNF-USA)}

\begin{table}[H]
\centering
\small
\begin{tabular}{|c|p{9.5cm}|c|}
\hline
\rowcolor{primaryBlue!20}
\textbf{ID} & \textbf{Descripción} & \textbf{Prioridad} \\
\hline
RNF-USA-01 & La interfaz debe ser completamente responsiva, adaptándose a dispositivos móviles, tablets y escritorio. & Alta \\
\hline
RNF-USA-02 & El sistema debe ser accesible según estándares WCAG 2.1 nivel AA. & Media \\
\hline
RNF-USA-03 & Los mensajes de error deben ser claros, descriptivos y orientados al usuario. & Alta \\
\hline
RNF-USA-04 & La navegación debe ser intuitiva, requiriendo máximo 3 clics para cualquier acción principal. & Media \\
\hline
RNF-USA-05 & El sistema debe proporcionar retroalimentación visual para todas las acciones del usuario. & Alta \\
\hline
\end{tabular}
\caption{Requerimientos No Funcionales de Usabilidad}
\end{table}

\subsection{Disponibilidad y Confiabilidad (RNF-AVL)}

\begin{table}[H]
\centering
\small
\begin{tabular}{|c|p{9.5cm}|c|}
\hline
\rowcolor{primaryBlue!20}
\textbf{ID} & \textbf{Descripción} & \textbf{Prioridad} \\
\hline
RNF-AVL-01 & El sistema debe tener una disponibilidad mínima del 99\% mensual. & Alta \\
\hline
RNF-AVL-02 & Las transacciones de base de datos deben cumplir propiedades ACID. & Alta \\
\hline
RNF-AVL-03 & El sistema debe implementar mecanismos de recuperación ante fallos de conexión. & Media \\
\hline
RNF-AVL-04 & Los datos deben respaldarse automáticamente de forma diaria. & Media \\
\hline
\end{tabular}
\caption{Requerimientos No Funcionales de Disponibilidad}
\end{table}

\subsection{Mantenibilidad (RNF-MNT)}

\begin{table}[H]
\centering
\small
\begin{tabular}{|c|p{9.5cm}|c|}
\hline
\rowcolor{primaryBlue!20}
\textbf{ID} & \textbf{Descripción} & \textbf{Prioridad} \\
\hline
RNF-MNT-01 & El código debe seguir convenciones de nomenclatura consistentes (camelCase, PascalCase según contexto). & Media \\
\hline
RNF-MNT-02 & La arquitectura debe seguir el patrón de separación de responsabilidades (MVC/MVP). & Alta \\
\hline
RNF-MNT-03 & El código debe estar documentado con comentarios significativos. & Media \\
\hline
RNF-MNT-04 & Las configuraciones sensibles deben manejarse mediante variables de entorno. & Alta \\
\hline
\end{tabular}
\caption{Requerimientos No Funcionales de Mantenibilidad}
\end{table}

% ============================================================================
\section{Historias de Usuario}

Las historias de usuario describen las funcionalidades del sistema desde la perspectiva del usuario final, siguiendo el formato estándar: ``Como [rol], quiero [funcionalidad], para [beneficio]''.

\subsection{Historias de Autenticación}

\noindent\fbox{\parbox{\dimexpr\textwidth-2\fboxsep-2\fboxrule}{
\textbf{HU-01: Registro de Usuario}\\[0.3cm]
\textbf{Como} usuario nuevo,\\
\textbf{Quiero} poder registrarme en el sistema proporcionando mi correo electrónico, contraseña y nombre,\\
\textbf{Para} poder acceder a las funcionalidades de gestión de prompts.\\[0.3cm]
\textbf{Criterios de Aceptación:}
\begin{itemize}[noitemsep,topsep=0pt]
    \item El formulario de registro solicita correo, contraseña y nombre.
    \item La contraseña debe tener mínimo 8 caracteres, una mayúscula y un número.
    \item Al registrarse, el usuario recibe un correo de verificación.
    \item Si el correo ya existe, se muestra un mensaje de error apropiado.
\end{itemize}
}}

\vspace{0.5cm}

\noindent\fbox{\parbox{\dimexpr\textwidth-2\fboxsep-2\fboxrule}{
\textbf{HU-02: Verificación de Correo}\\[0.3cm]
\textbf{Como} usuario registrado,\\
\textbf{Quiero} verificar mi correo electrónico mediante un enlace enviado a mi bandeja,\\
\textbf{Para} confirmar que tengo acceso al correo proporcionado y activar mi cuenta.\\[0.3cm]
\textbf{Criterios de Aceptación:}
\begin{itemize}[noitemsep,topsep=0pt]
    \item El correo de verificación contiene un enlace único y temporal.
    \item Al hacer clic en el enlace, la cuenta se activa.
    \item Si el enlace expira, el usuario puede solicitar uno nuevo.
    \item El sistema muestra confirmación de verificación exitosa.
\end{itemize}
}}

\vspace{0.5cm}

\noindent\fbox{\parbox{\dimexpr\textwidth-2\fboxsep-2\fboxrule}{
\textbf{HU-03: Inicio de Sesión}\\[0.3cm]
\textbf{Como} usuario registrado,\\
\textbf{Quiero} poder iniciar sesión con mi correo y contraseña,\\
\textbf{Para} acceder a mi colección personal de prompts.\\[0.3cm]
\textbf{Criterios de Aceptación:}
\begin{itemize}[noitemsep,topsep=0pt]
    \item El formulario de login solicita correo y contraseña.
    \item Las credenciales correctas permiten acceso al dashboard.
    \item Las credenciales incorrectas muestran un mensaje de error genérico.
    \item Después de 5 intentos fallidos, se bloquea temporalmente el acceso.
\end{itemize}
}}

\subsection{Historias de Gestión de Prompts}

\noindent\fbox{\parbox{\dimexpr\textwidth-2\fboxsep-2\fboxrule}{
\textbf{HU-04: Crear Prompt}\\[0.3cm]
\textbf{Como} usuario autenticado,\\
\textbf{Quiero} poder crear un nuevo prompt con título, contenido y etiquetas,\\
\textbf{Para} almacenar y organizar mis prompts de IA.\\[0.3cm]
\textbf{Criterios de Aceptación:}
\begin{itemize}[noitemsep,topsep=0pt]
    \item El formulario incluye campos para título (obligatorio), contenido (obligatorio), descripción (opcional) y etiquetas (opcional).
    \item El título tiene un límite de 200 caracteres.
    \item Se pueden agregar hasta 10 etiquetas por prompt.
    \item Al guardar, el prompt aparece inmediatamente en la lista.
\end{itemize}
}}

\vspace{0.5cm}

\noindent\fbox{\parbox{\dimexpr\textwidth-2\fboxsep-2\fboxrule}{
\textbf{HU-05: Visualizar Lista de Prompts}\\[0.3cm]
\textbf{Como} usuario autenticado,\\
\textbf{Quiero} ver una lista paginada de todos mis prompts,\\
\textbf{Para} tener una visión general de mi colección.\\[0.3cm]
\textbf{Criterios de Aceptación:}
\begin{itemize}[noitemsep,topsep=0pt]
    \item La lista muestra título, descripción resumida y etiquetas de cada prompt.
    \item La lista está paginada con 10 elementos por página por defecto.
    \item Se muestra la fecha de última modificación de cada prompt.
    \item La lista puede ordenarse por fecha de creación o modificación.
\end{itemize}
}}

\vspace{0.5cm}

\noindent\fbox{\parbox{\dimexpr\textwidth-2\fboxsep-2\fboxrule}{
\textbf{HU-06: Editar Prompt}\\[0.3cm]
\textbf{Como} usuario autenticado,\\
\textbf{Quiero} poder editar un prompt existente,\\
\textbf{Para} actualizar o mejorar su contenido.\\[0.3cm]
\textbf{Criterios de Aceptación:}
\begin{itemize}[noitemsep,topsep=0pt]
    \item Todos los campos del prompt son editables.
    \item Los cambios se guardan y la fecha de modificación se actualiza.
    \item Si otro proceso modificó el prompt, se notifica del conflicto.
    \item El usuario puede cancelar la edición sin guardar cambios.
\end{itemize}
}}

\vspace{0.5cm}

\noindent\fbox{\parbox{\dimexpr\textwidth-2\fboxsep-2\fboxrule}{
\textbf{HU-07: Eliminar Prompt}\\[0.3cm]
\textbf{Como} usuario autenticado,\\
\textbf{Quiero} poder eliminar un prompt que ya no necesito,\\
\textbf{Para} mantener mi colección organizada.\\[0.3cm]
\textbf{Criterios de Aceptación:}
\begin{itemize}[noitemsep,topsep=0pt]
    \item El sistema solicita confirmación antes de eliminar.
    \item La eliminación es permanente e irreversible.
    \item Tras eliminar, el prompt desaparece de la lista inmediatamente.
    \item Se muestra un mensaje de confirmación exitosa.
\end{itemize}
}}

\subsection{Historias de Búsqueda y Filtrado}

\noindent\fbox{\parbox{\dimexpr\textwidth-2\fboxsep-2\fboxrule}{
\textbf{HU-08: Buscar Prompts}\\[0.3cm]
\textbf{Como} usuario autenticado,\\
\textbf{Quiero} poder buscar prompts por palabras clave,\\
\textbf{Para} encontrar rápidamente prompts específicos.\\[0.3cm]
\textbf{Criterios de Aceptación:}
\begin{itemize}[noitemsep,topsep=0pt]
    \item La búsqueda analiza título, contenido y descripción.
    \item Los resultados se muestran en tiempo real mientras se escribe.
    \item Se resaltan las coincidencias en los resultados.
    \item Si no hay resultados, se muestra un mensaje apropiado.
\end{itemize}
}}

\vspace{0.5cm}

\noindent\fbox{\parbox{\dimexpr\textwidth-2\fboxsep-2\fboxrule}{
\textbf{HU-09: Filtrar por Etiquetas}\\[0.3cm]
\textbf{Como} usuario autenticado,\\
\textbf{Quiero} poder filtrar mis prompts por etiquetas,\\
\textbf{Para} ver solo prompts de una categoría específica.\\[0.3cm]
\textbf{Criterios de Aceptación:}
\begin{itemize}[noitemsep,topsep=0pt]
    \item Se muestra una lista de todas las etiquetas disponibles.
    \item Al seleccionar una etiqueta, solo se muestran prompts con esa etiqueta.
    \item Se pueden combinar múltiples etiquetas como filtro.
    \item El filtro se puede limpiar para ver todos los prompts.
\end{itemize}
}}

\subsection{Historias de Exportación}

\noindent\fbox{\parbox{\dimexpr\textwidth-2\fboxsep-2\fboxrule}{
\textbf{HU-10: Exportar Prompts}\\[0.3cm]
\textbf{Como} usuario autenticado,\\
\textbf{Quiero} poder exportar mis prompts en diferentes formatos,\\
\textbf{Para} crear respaldos o compartir mi colección.\\[0.3cm]
\textbf{Criterios de Aceptación:}
\begin{itemize}[noitemsep,topsep=0pt]
    \item Se ofrecen formatos JSON, Markdown, CSV y TXT.
    \item El usuario puede exportar toda la colección o prompts seleccionados.
    \item El archivo se descarga automáticamente en el navegador.
    \item El nombre del archivo incluye la fecha de exportación.
\end{itemize}
}}

% ============================================================================
\section{Análisis del Proyecto}

Esta sección presenta el análisis técnico del proyecto, incluyendo la definición del stack tecnológico, la arquitectura del sistema y las consideraciones de diseño.

\subsection{Stack Tecnológico}

El stack tecnológico seleccionado para el proyecto ``Bóveda de Prompts'' se basa en tecnologías modernas, robustas y ampliamente adoptadas en la industria.

\subsubsection{Tecnologías del Frontend}

\begin{table}[H]
\centering
\begin{tabular}{|l|l|p{7cm}|}
\hline
\rowcolor{primaryBlue!20}
\textbf{Tecnología} & \textbf{Versión} & \textbf{Justificación} \\
\hline
React & 18.x & Biblioteca líder para construcción de interfaces de usuario con arquitectura basada en componentes. \\
\hline
Vite & 5.x & Herramienta de construcción moderna con tiempos de compilación rápidos y Hot Module Replacement. \\
\hline
Tailwind CSS & 4.x & Framework CSS utilitario para desarrollo rápido con diseños consistentes. \\
\hline
React Router & 6.x & Librería estándar para manejo de rutas en aplicaciones React SPA. \\
\hline
Axios & 1.x & Cliente HTTP robusto con soporte para interceptores. \\
\hline
React Hook Form & 7.x & Librería para manejo de formularios con validación eficiente. \\
\hline
Zod & 3.x & Librería de validación de esquemas TypeScript-first. \\
\hline
Lucide React & - & Conjunto de íconos SVG optimizados. \\
\hline
\end{tabular}
\caption{Stack Tecnológico del Frontend}
\end{table}

\subsubsection{Tecnologías del Backend}

\begin{table}[H]
\centering
\begin{tabular}{|l|l|p{7cm}|}
\hline
\rowcolor{primaryBlue!20}
\textbf{Tecnología} & \textbf{Versión} & \textbf{Justificación} \\
\hline
Node.js & 18+ LTS & Entorno de ejecución JavaScript con excelente rendimiento para operaciones I/O. \\
\hline
Express.js & 4.x & Framework web minimalista y flexible para construcción de APIs RESTful. \\
\hline
PostgreSQL & 14+ & Base de datos relacional robusta con soporte para búsqueda full-text y transacciones ACID. \\
\hline
JWT & - & Estándar para tokens de autenticación stateless. \\
\hline
Bcrypt & - & Algoritmo de hash adaptativo para almacenamiento seguro de contraseñas. \\
\hline
NodeMailer & - & Módulo para envío de correos electrónicos. \\
\hline
Helmet & - & Middleware de seguridad para headers HTTP. \\
\hline
\end{tabular}
\caption{Stack Tecnológico del Backend}
\end{table}

\subsection{Arquitectura del Sistema}

El sistema ``Bóveda de Prompts'' implementa una arquitectura cliente-servidor de tres capas.

\subsubsection{Vista de Alto Nivel}

La arquitectura se divide en tres capas principales:

\begin{enumerate}
    \item \textbf{Capa de Presentación (Frontend)}: Aplicación React SPA que se ejecuta en el navegador del usuario.
    
    \item \textbf{Capa de Lógica de Negocio (Backend)}: API RESTful en Express.js que procesa las solicitudes y aplica reglas de negocio.
    
    \item \textbf{Capa de Datos (Base de Datos)}: PostgreSQL almacena y gestiona todos los datos persistentes.
\end{enumerate}

\subsubsection{Estructura del Backend}

El backend sigue una arquitectura modular organizada por responsabilidades:

\begin{verbatim}
servidor/
  src/
    config/          # Configuración (BD, email, env)
    controladores/   # Lógica de negocio por módulo
    middleware/      # Middlewares (auth, validación)
    rutas/           # Definición de endpoints API
    servicios/       # Servicios (email, externo)
    utilidades/      # Utilidades (exportación, helpers)
    servidor.js      # Punto de entrada
  package.json
\end{verbatim}

\subsubsection{Estructura del Frontend}

El frontend sigue una arquitectura basada en componentes:

\begin{verbatim}
cliente/
  src/
    componentes/     # Componentes reutilizables
      comunes/       # Botón, Input, Modal, etc.
      autenticacion/
      prompts/
      layout/
    contexto/        # Context API (estado global)
    paginas/         # Vistas/páginas principales
    servicios/       # Servicios de comunicación API
    main.jsx         # Punto de entrada
  package.json
\end{verbatim}

\subsection{Diseño de la Base de Datos}

El modelo de datos está normalizado y optimizado para las operaciones más frecuentes del sistema.

\subsubsection{Modelo Entidad-Relación}

El sistema maneja las siguientes entidades principales:

\begin{itemize}
    \item \textbf{Usuarios}: Almacena información de cuentas de usuario, credenciales y estado de verificación.
    
    \item \textbf{Prompts}: Almacena los prompts con su contenido, metadatos y relación con el usuario propietario.
    
    \item \textbf{Categorías} (opcional): Taxonomía predefinida para clasificación de prompts.
\end{itemize}

\subsubsection{Esquema de Tablas}

\textbf{Tabla usuarios:}
\begin{itemize}[noitemsep]
    \item id (SERIAL PRIMARY KEY)
    \item email (VARCHAR UNIQUE NOT NULL)
    \item contrasena\_hash (VARCHAR NOT NULL)
    \item nombre (VARCHAR NOT NULL)
    \item email\_verificado (BOOLEAN DEFAULT FALSE)
    \item token\_verificacion (VARCHAR)
    \item creado\_en (TIMESTAMP DEFAULT NOW())
    \item actualizado\_en (TIMESTAMP DEFAULT NOW())
\end{itemize}

\textbf{Tabla prompts:}
\begin{itemize}[noitemsep]
    \item id (SERIAL PRIMARY KEY)
    \item usuario\_id (INTEGER REFERENCES usuarios)
    \item titulo (VARCHAR NOT NULL)
    \item contenido (TEXT NOT NULL)
    \item descripcion (TEXT)
    \item etiquetas (TEXT[])
    \item es\_favorito (BOOLEAN DEFAULT FALSE)
    \item creado\_en (TIMESTAMP DEFAULT NOW())
    \item actualizado\_en (TIMESTAMP DEFAULT NOW())
\end{itemize}

\subsubsection{Índices y Optimizaciones}

Se implementan índices para optimizar las consultas más frecuentes:

\begin{itemize}
    \item Índice único en usuarios(email)
    \item Índice en prompts(usuario\_id)
    \item Índice en prompts(creado\_en)
    \item Índice GIN en prompts(etiquetas) para búsqueda de arreglos
    \item Índice GIN para búsqueda full-text en título y contenido
\end{itemize}

\subsection{Diseño de la API RESTful}

La API sigue los principios REST y utiliza códigos de estado HTTP estándar.

\subsubsection{Endpoints Principales}

\begin{table}[H]
\centering
\small
\begin{tabular}{|l|l|p{6cm}|}
\hline
\rowcolor{primaryBlue!20}
\textbf{Método} & \textbf{Endpoint} & \textbf{Descripción} \\
\hline
POST & /api/auth/registrar & Registro de nuevo usuario \\
\hline
POST & /api/auth/iniciar-sesion & Inicio de sesión \\
\hline
POST & /api/auth/verificar-email & Verificación de correo \\
\hline
GET & /api/auth/yo & Obtener usuario actual \\
\hline
GET & /api/prompts & Listar prompts (paginado) \\
\hline
GET & /api/prompts/:id & Obtener prompt por ID \\
\hline
POST & /api/prompts & Crear nuevo prompt \\
\hline
PUT & /api/prompts/:id & Actualizar prompt \\
\hline
DELETE & /api/prompts/:id & Eliminar prompt \\
\hline
GET & /api/prompts/estadisticas & Obtener estadísticas \\
\hline
GET & /api/exportar & Exportar prompts \\
\hline
\end{tabular}
\caption{Endpoints de la API REST}
\end{table}

\subsubsection{Formato de Respuestas}

Todas las respuestas siguen un formato JSON estandarizado:

\textbf{Respuesta exitosa:}
\begin{verbatim}
{
    "mensaje": "Operación exitosa",
    "datos": { ... }
}
\end{verbatim}

\textbf{Respuesta de error:}
\begin{verbatim}
{
    "error": "CODIGO_ERROR",
    "mensaje": "Descripción del error",
    "accion": "ACCION_SUGERIDA"
}
\end{verbatim}

\subsection{Consideraciones de Seguridad}

El diseño del sistema incorpora múltiples capas de seguridad:

\begin{enumerate}
    \item \textbf{Autenticación}: JWT con expiración configurable y tokens de refresh.
    
    \item \textbf{Autorización}: Middleware que verifica permisos antes de cada operación.
    
    \item \textbf{Protección de datos}: Hash bcrypt para contraseñas con factor 12.
    
    \item \textbf{Prevención de ataques}:
    \begin{itemize}
        \item Rate limiting para prevenir fuerza bruta
        \item Queries parametrizados contra SQL injection
        \item Headers de seguridad con Helmet
        \item Validación de entrada en frontend y backend
    \end{itemize}
    
    \item \textbf{Comunicación segura}: HTTPS obligatorio en producción.
\end{enumerate}

\subsection{Flujo de Trabajo del Usuario}

El flujo típico de un usuario en el sistema es:

\begin{enumerate}
    \item El usuario accede a la página de inicio.
    \item Se registra proporcionando correo, contraseña y nombre.
    \item Recibe y confirma el correo de verificación.
    \item Inicia sesión con sus credenciales.
    \item Accede al dashboard donde ve sus prompts.
    \item Puede crear, editar, buscar, filtrar y eliminar prompts.
    \item Puede exportar su colección en el formato deseado.
    \item Cierra sesión cuando termina.
\end{enumerate}

\subsection{Plan de Implementación}

El desarrollo del proyecto se organiza en las siguientes fases:

\begin{enumerate}
    \item \textbf{Fase 1 - Configuración}:
    \begin{itemize}
        \item Configuración del entorno de desarrollo
        \item Inicialización de proyectos frontend y backend
        \item Configuración de base de datos
    \end{itemize}
    
    \item \textbf{Fase 2 - Backend}:
    \begin{itemize}
        \item Desarrollo del módulo de autenticación
        \item Desarrollo del módulo de prompts
        \item Desarrollo del módulo de exportación
        \item Implementación de seguridad y validaciones
    \end{itemize}
    
    \item \textbf{Fase 3 - Frontend}:
    \begin{itemize}
        \item Desarrollo de componentes comunes
        \item Desarrollo de páginas de autenticación
        \item Desarrollo de páginas de gestión de prompts
        \item Integración con backend
    \end{itemize}
    
    \item \textbf{Fase 4 - Pruebas y Despliegue}:
    \begin{itemize}
        \item Pruebas unitarias e integración
        \item Pruebas de usuario
        \item Configuración de producción
        \item Despliegue inicial
    \end{itemize}
\end{enumerate}

% ============================================================================
\section{Conclusiones}

El presente informe documenta el levantamiento de requerimientos para el sistema ``Bóveda de Prompts'', una aplicación web diseñada para resolver el problema de la gestión desorganizada de prompts de inteligencia artificial.

A través de este análisis se han identificado:

\begin{itemize}
    \item \textbf{30 requerimientos funcionales} organizados en 5 módulos: Autenticación, Gestión de Prompts, Búsqueda y Filtrado, Exportación, y Estadísticas.
    
    \item \textbf{20 requerimientos no funcionales} que cubren aspectos de seguridad, rendimiento, usabilidad, disponibilidad y mantenibilidad.
    
    \item \textbf{10 historias de usuario} que describen las funcionalidades desde la perspectiva del usuario final.
\end{itemize}

El stack tecnológico seleccionado (React, Node.js, Express, PostgreSQL) proporciona una base sólida y escalable para el desarrollo del sistema, mientras que la arquitectura de tres capas garantiza la separación de responsabilidades y la mantenibilidad del código.

El proyecto presenta un enfoque integral que abarca desde la definición de requerimientos hasta el plan de implementación, sentando las bases para un desarrollo exitoso del sistema.

\end{document}
